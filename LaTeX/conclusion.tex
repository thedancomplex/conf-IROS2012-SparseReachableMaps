\section{CONCLUSION AND FUTURE WORK}\label{sec:conc}

As the results in Fig.~\ref{fig:sparseRegion} in Section~\ref{sec:reslts} show the approach presented in this paper was successful for underhand throwing.  This approach also preforms overhand throwing if the velocity direction, duration, and magnitude falls within the SRM.  While not presented in this paper, results of overhand throwing will be presented in future dissemination of this work.  This work is also constructed in such a way that it is easily applicable to other low and high DOF robots.

The paper described a solution that coincides with our future efforts.  The next logical step is to incorporate full body motion and balancing to the velocity trajectory calculations to further advance the overarching goal of full body end-effector velocity control.

%In the end we solved the problem in the direction that we are going in.  The next logical step is to incorporate full body motion and balancing to the velocity trajectory calculations to further advance us towards our overarching goal.


%This work has shown a valid method of creating trajectories to achieve end-effector velocity control for high degree of freedom position controlled robots.  It was shown the full reachable area does not need to be known to achieve the desired velocity if a good collision model of the robot is available.  It was found that the limiting factor was the robot's physical joints.  This system does create trajectories that fall within the actuators' limitations, however this is not guaranteed.  Immediate future work includes creating a system that will guarantee actuator compliance with the generated trajectory.