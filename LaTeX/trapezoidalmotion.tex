\subsection{On-Line Trapezoidal Motion Profile}\label{sec:trap}

The robot's starting position $x_0$ is not guaranteed to be the same as the
first point in the velocity trajectory $L_d$.  To avoid over large
accelerations when giving this step input from $x_0$ to $L_d(0)$ an on-line
trapezoidal motion profile (TMP) was used to generate joint space commands with
the desired limited angular acceleration and velocity.  The TMP was only active
during the setup phase where the robot's end-effector moves from $x_0$ to
$L_d(0)$.  This is because the TMP's inherent nature has the potential to adversely effect the desired velocity in $R^3$ under high angular velocity and acceleration conditions in joint space.

The TMP was designed to limit the applied angular velocity and acceleration in
joint space and to prevent over-current/torque. An important advantage over
simply limiting output velocity and acceleration is that the TMP has little to
no overshoot. When a clipped and rate-limited velocity profile is integrated,
the resulting position trajectory may over or undershoot due to this non-linear
system behavior.  The TMP accounts for the imposed limits inherently, and will
arrive at a static goal without overshoot.
%TODO: Several statements in here are a bit conjectural

There are three major steps from the initial state to a goal position:
%(1) Accelerate at maximum acceleration in the direction of the goal
%(2) Achieve and hold maximum velocity
%(3) Decelerate to zero velocity to reach goal

\begin{enumerate}
\item Accelerate at maximum acceleration in the direction of the goal
\item Achieve and hold maximum velocity
\item Decelerate to zero velocity to reach goal
\end{enumerate}

The area under the velocity trapezoid is the total displacement achieved by the
profile. By shaping this profile based on initial and goal conditions, any
goal position can be precisely reached, even if velocity clipping occurs. The
shape of the profile can be challenging to identify, however, since it is not
always a trapezoid. For small displacements and large velocity / acceleration
limits, the profile will only reach a fraction of maximum velocity, and will be
triangular. The varying shape of the profile means that calculating and storing
the entire profile for each update could be computationally expensive. 
%THis is a weak statement, since I can't back that up.  Really, it's just unncecessary since my way is better.

The area under the trapezoid can be divided into an accelerating
\eqref{eq:trapslope} and constant velocity portion, where the area is simply a
triangle. the time to accelerate is solved from the final velocity and the
acceleration limit.

\begin{equation}
\label{eq:trapslope}
\Delta x=\frac{v_{max}\tau}{2}=\frac{v_{max}^2 sign(v_{max})}{2a_{max}}
\end{equation}

An important consequence of this calculation is that all trapezoidal profiles
end by decelerating at the acceleration limit to the goal position. Rewriting
\eqref{eq:trapslope} slightly, the deceleration (stopping) distance $d_s$ can be found
from the current velocity $v_0$ in \eqref{eq:decdist}. As long as the goal
distance and deceleration distance are equal, then the controller simply
needs to decelerate at the maximum rate to come to rest at the goal.
Conversely, for the current goal distance, there is a critical velocity $v_c$
such that, if the joint began moving at this velocity in the following
timestep, it could decelerate at the maximum rate to reach the position goal.
The controller's action is simply to minimize the error between $v_c$ and $v$,
the output velocity of the controller at each timestep.

Since the joint is moving with velocity $v_0$ during a current time-step, some
initial distance $x_{err}$ \eqref{eq:xerr} is travelled before the joint can be
affected. Defining $\hat{u}$ as the sign of the distance to the goal, $v_c$ is
related to $x_g$ and $x_{err}$ quadratically in \eqref{eq:vc2}. This equation
assumes simple trapezoidal integration. Solving for $v_c$ using the quadratic
formula produces in general complex roots due to the possibility of negative
$v_0$ or $x_g$.  In \eqref{eq:vcrit}, $v_0*\hat{u}$ is the current velocity
relative to the goal direction, producing a positive term if the signs of both
terms match. This result will always produce a real value for all $v_0$ and
$x_g$.
%TODO: Prove this? it should be possible to plot the root as a surface over v_0
%and x_g

\begin{equation}
\label{eq:decdist}
d_s=\frac{v_0^2 sign(v_0)}{2 a_{max}}
\end{equation}

\begin{equation}
\label{eq:xerr}
x_{err}=v_0 \tau + \frac{v_c-v_0}{2}\tau
\end{equation}

\begin{equation}
\label{eq:uhat}
\hat{u}=sign(x_g)
\end{equation}

%\begin{equation}
%\label{eq:vc1}
%v_s^2=x_g 2 a_{max}
%\end{equation}

\begin{equation}
\label{eq:vc2}
v_c^2 = 2 a_{max} \left(x_g -x_{err} \right)
\end{equation}

\begin{equation}
\label{eq:vcrit}
v_c=\hat{u} a_{max} \left(\sqrt{\frac{a_{max} \tau^2 - \hat{u} 4 v_0 \tau+8 |x_g|}{4 a_{max}}} - \frac{ \tau }{2}\right)
\end{equation}
